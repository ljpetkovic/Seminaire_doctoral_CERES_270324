\begin{frame}{Première analyse du corpus Charcot}
\color{deepblue}{OBVIE\footnote{\url{https://obtic.huma-num.fr/obvie/}}}
\begin{itemize}
    \item moteur de recherche pour la fouille avancée des corpus en \textsc{XML-TEI}
    \item identification des substantifs les plus importants 
    \begin{itemize}
        \item fréquences brutes, mesures \textsc{TF-IDF}, \textsc{BM25}, $\chi^2$, Test Gamma
    \end{itemize}
    \item repérage des textes similaires par ordre de pertinence
    \begin{itemize}
    \item à partir des termes en commun et termes fréquents
\end{itemize}     
\end{itemize}
\end{frame}



\begin{frame}{Deuxième analyse du corpus Charcot}
    \color{deepblue}{TextPair\footnote{\scriptsize\url{https://artfl-project.uchicago.edu/text-pair}}}
    \begin{itemize}
        \item aligne des passages similaires dans une collection de textes
        \begin{itemize}
        \item passages incluant des citations, plagiats, emprunts, réemplois$\dots$
        \end{itemize}
        \item génère une liste de séquences similaires pour chaque texte
        \begin{itemize}
        \item séquences de mots qui se chevauchent (trigrammes de mots)
        \end{itemize}        
        \item compare les séquences générées du texte \textit{source} au texte \textit{cible}
    \end{itemize}
\end{frame}



\begin{frame}{Vers une lecture plus distante du corpus Charcot}
Analyse de textes assistée par ordinateur
    \begin{itemize}
        \item recherche avancée et alignement de textes (OBVIE, TextPair)
        \begingroup 
\setbeamertemplate{itemize items}{}
\endgroup
        \item manque de fonctionnalités pour mesurer l’impact de Charcot sur son réseau \textit{via} les concepts de ses travaux médicaux 
        \begin{itemize}
        \item[$\rightarrow$] recherche d'un outil de \og{}lecture distante\fg{}
        \end{itemize}
        % \item Inconvénients : fonctionnalités de lecture distante permettant de rendre compte de l’impact de Charcot sur son réseau scientifique et artistique à travers les concepts principaux de ses travaux
    \end{itemize}
    \begin{block}{Une nouvelle approche}
        \begin{itemize}
    \item quantification de la pertinence des concepts polylexicaux dans les corpus, selon trois différentes métriques de pondération
    \item alignement des mots-clés issus de deux corpus (validation auprès de
spécialistes de Charcot nécessaire)
\end{itemize}
    \end{block}
% \textbf{Une nouvelle approche}
% \begin{itemize}
%     \item quantification de la pertinence des concepts polylexicaux dans les corpus, selon trois différentes métriques de pondération
%     \item repérage des phénomènes lexicaux grâce aux visualisations (validation auprès de
% spécialistes de Charcot nécessaire)
% \end{itemize}
\end{frame}

\begin{frame}{Perspectives}
    \begin{enumerate}
        \item Charcot vs. Autres : initateur ou transmetteur de certains termes ?
        \item analyse sémantique des passages contenant ces concepts $\rightarrow{}$ modalités de prise en charge énonciative
        \begin{itemize}
            \item opinions,
accords, désaccords, définitions, etc.
        \end{itemize}
        \item \textsc{OCR}iser les manuscrits (\og{}leçons\fg{}) de Charcot avec eScriptorium\footnote{\url{https://escriptorium.isir.upmc.fr/}}
        \item post-correction automatique d'\textsc{OCR} avec la librairie \texttt{neuspell}
		\begin{flushright}
		\vspace{-0.2cm}
	{\footnotesize\citep{jayanthi2020neuspell}}	
	\end{flushright}		       
        \begin{itemize}
        \item évaluation de son impact sur des tâches en aval 
        \end{itemize}
        \item modélisation de sujets assistée par l'apprentissage automatique
        \begin{flushright}
        \vspace{-0.2cm}
        {\footnotesize\citep{grootendorst2022bertopic}}
		\end{flushright}         
    \end{enumerate}
\end{frame}

\begin{frame}{Données et scripts}
Dépôts GitHub :
\begin{itemize}
\item \href{https://github.com/ljpetkovic/Seminaire_doctoral_CERES_270324/settings}{Diapositives}
\item \href{https://github.com/ljpetkovic/Charcot_circulations}{Tracking the circulation of Jean-Martin Charcot’s medical discours$\dots$}
\item passage à l'échelle à l'aide du supercalculateur \textsc{SACADO}\footnote{\url{https://sacado.sorbonne-universite.fr/}.}
\item \href{https://github.com/ljpetkovic Charcot_KeyBERT_Keyphrase-Vectorizers}{Extraction des mots-clés à partir des textes}
\end{itemize}
\end{frame}

\begin{frame}{Remerciements}
\justifying
Un grand merci à Valentina Fedchenko (ingénieure de recherche de l'équipe-projet ObTIC) et à Simon Gabay (maître-assistant de la Chaire des humanités numériques à l'Unige) pour leurs conseils précieux.
\end{frame}
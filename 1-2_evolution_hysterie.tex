\begin{frame}{Évolution du terme \textit{hystérie}}
\begin{table}[]
\begin{tabular}{lcl}
\multicolumn{1}{l}{\textrm{Période}} & \multicolumn{1}{c}{\textrm{Sexe}} & \multicolumn{1}{c}{\textrm{Étiologie}}   \\
\hline \hline
Antiquité   & \female{}             & \begin{tabular}[c]{@{}l@{}}{\small déplacement de l'utérus, selon Hippocrate\footnote{{\small\cite{tasca2012women}}}}\\\end{tabular}          \\ \hline
Moyen Âge   & \female{}             & \begin{tabular}[c]{@{}l@{}}{\small possession démoniaque (superstition religieuse)\footnote{{\small\cite{tasca2012women}}}}\end{tabular} \\ \hline
Renaissance & \female{}/\male{}           & \begin{tabular}[c]{@{}l@{}}{\small localisation dans le \textit{sensorium commune}\footnote{{\small\cite{lepois1618}}}}\end{tabular}   \\ \hline
Lumières    & \female{}/\male{}           & \begin{tabular}[c]{@{}l@{}}{\small explosion des \og{}esprits animaux\fg{} dans le cerveau\footnote{{\small\cite{willispathologiae}}}}\end{tabular} \\ \hline
\rowcolor{LightCyan}
\textsc{XIX}\ieme{} s. & \female{}/\male{} & \begin{tabular}[c]{@{}l@{}}{\small dégénérescence héréditaire du système nerveux\footnote{{\small\cite{charcot1870}}}}\end{tabular}
\end{tabular}
\end{table}
\end{frame}




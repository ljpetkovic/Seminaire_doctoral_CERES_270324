\begin{frame}{Évolution du terme \textit{hystérie}}
Exemple du changement de paradigme : le terme d'\textsc{hystérie}
\begin{itemize}
 \item gr. \textgreek{ὑστέρα}, lat. \textit{hystera} : \leftguillemet{} utérus \rightguillemet{}, \leftguillemet{} matrice \rightguillemet{}
\end{itemize}
\begin{table}[h!]
\footnotesize
\begin{tabular}{lcl}
\multicolumn{1}{l}{\textrm{Période}} & \multicolumn{1}{c}{\textrm{Sexe}} & \multicolumn{1}{c}{\textrm{Étiologie}}   \\
\hline \hline
Antiquité   & \female{}             & \begin{tabular}[c]{@{}l@{}}{\footnotesize déplacement de l'utérus, selon Hippocrate \citep{tasca2012women}
%\footnote{{\footnotesize\cite{tasca2012women}}}
\\\textit{hystérique} : (femme) malade de l'utérus}\end{tabular}          \\ \hline
Moyen Âge   & \female{}             & \begin{tabular}[c]{@{}l@{}}{\footnotesize possession démoniaque, superstition religieuse de l'Église
%\footnote{{\footnotesize\cite{tasca2012women}}}
\\ 	$\rightarrow$ chasses, tortures, exorcismes \citep{tasca2012women}}\end{tabular} \\ \hline
Renaissance & \female{}/\male{}           & \begin{tabular}[c]{@{}l@{}}{\footnotesize localisation dans le cerveau, \textit{sensorium commune} \citep{lepois1618}
%\footnote{{\footnotesize\cite{lepois1618}}}
\\\leftguillemet{} siège commun de la sensibilité \rightguillemet{}\footnote{\cite{kant1863}.}, ensemble des perceptions}\end{tabular}   \\ \hline
Lumières    & \female{}/\male{}           & \begin{tabular}[c]{@{}l@{}}{\footnotesize explosion des \leftguillemet{} esprits animaux \rightguillemet{} dans le cerveau\\maladie convulsive \citep{willispathologiae}\footnote{{\footnotesize créateur du terme \textit{neurologia} en 1664.}}}\end{tabular} \\ \hline
\rowcolor{LightCyan}
\textsc{XIX}\ieme{} s. & \female{}/\male{} & \begin{tabular}[c]{@{}l@{}}{\footnotesize dégénérescence héréditaire du système nerveux \citep{charcot1870}
%\footnote{{\footnotesize\cite{charcot1870}}}
\\maladie systématiquement traitée comme un trouble neurologique}
\end{tabular}
\end{tabular}
\end{table}
\begin{itemize}
\end{itemize}
\end{frame}




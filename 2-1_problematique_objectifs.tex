\begin{frame}{Impact de Charcot sur sa discipline et au-delà}
\textbf{Collaborateurs, élèves, savants polymathes}
\centering
    \begin{table}[!ht]
        \centering
        \begin{tabular}{l r}
           Sigmund \textsc{Freud} (1856-1939)  & théorie psychanalytique \\
            Gilles \textsc{de la Tourette} (1857-1932) & syndrôme de Tourette \\
            Joseph \textsc{Babinski} (1857-1904) & signe de Babinski \\
        \end{tabular}
        % \caption{Caption}
        \label{tab:my_label}
    \end{table}
\medskip
\textbf{Littéraires} (\cite{koehler2013charcot}) 
\begin{itemize}
\item références à Charcot et descriptions de crises hystériques dans la littérature naturaliste française et européenne
\end{itemize}
\begin{table}[!ht]
    \centering
    \begin{tabular}{l r}
        Émile \textsc{Zola} (1840–1902)  & \textit{Lourdes} \\
        Léon \textsc{Tolstoï} (1828–1910) & \textit{La Sonate à Kreutzer} \\
        Luigi \textsc{Capuana} (1839–1915) & \textit{La Torture}
    \end{tabular}
    % \caption{Caption}
    \label{tab:my_label}
\end{table}

\end{frame}


\begin{frame}{Vers une lecture plus distante du corpus Charcot}
\textbf{Premières explorations du corpus Charcot}
    \begin{itemize}
        \item recherche avancée et alignement de textes $\rightarrow$ analyse de textes assistée par ordinateur (OBVIE, TextPair)
        \item manque de fonctionnalités pour mesurer l’impact de Charcot sur son réseau \textit{via} les concepts principaux de ses travaux $\rightarrow$ lecture distante
        % \item Inconvénients : fonctionnalités de lecture distante permettant de rendre compte de l’impact de Charcot sur son réseau scientifique et artistique à travers les concepts principaux de ses travaux
    \end{itemize}
    \begin{block}{Une nouvelle approche}
        \begin{itemize}
    \item quantification de la pertinence des concepts polylexicaux dans les corpus, selon trois différentes métriques de pondération
    \item repérage des phénomènes lexicaux grâce aux visualisations (validation auprès de
spécialistes de Charcot nécessaire)
\end{itemize}
    \end{block}
% \textbf{Une nouvelle approche}
% \begin{itemize}
%     \item quantification de la pertinence des concepts polylexicaux dans les corpus, selon trois différentes métriques de pondération
%     \item repérage des phénomènes lexicaux grâce aux visualisations (validation auprès de
% spécialistes de Charcot nécessaire)
% \end{itemize}
\end{frame}

\begin{frame}{Perspectives}
\textbf{Recherches futures}
    \begin{enumerate}
        \item Charcot vs. Autres : initateur ou transmetteur de certains termes ?
        \item analyse sémantique des passages contenant ces concepts $\rightarrow{}$ modalités de prise en charge énonciative
        \begin{itemize}
            \item opinions,
accords, désaccords, définitions, etc.
        \end{itemize}
        \item post-correction automatique d'OCR (apprentissage profond) et évaluation de son impact sur des tâches en aval
        \item modélisation de sujets dynamiques\footnote{\textit{cf.} \cite{blei2006}.} pour pister l'évolution diachronique des concepts de Charcot
    \end{enumerate}
\end{frame}

\begin{frame}{Données et scripts}
Dépôt GitHub : \url{https://github.com/ljpetkovic/Charcot_circulations}
\end{frame}

\begin{frame}{Remerciements}
Un grand merci à Valentina Fedchenko (ingénieure de recherche de l'équipe-projet ObTIC) et à Simon Gabay (maître-assistant de la Chaire des humanités numériques à l'Unige) pour leurs conseils précieux.
\end{frame}